\chapter{Introducere}

\section{Motivația lucrării}
\subsection{Problemă}
În cadrul firmei în care am efectuat practica vara trecută, prezența la birou mi s-a părut utilă pentru facilitarea colaborării în cadrul echipei de proiect, însă acest lucru venea și cu dezavantaje. Gândul că exista posibilitatea să nu găsesc un birou liber lângă cei de la care învățam mă neliniștea întrucât relaționarea și comunicarea nu mai erau oportune. Din cauza acestei probleme, încercam să ajung dimineața cât mai devreme astfel încât să găsesc un loc liber lângă cei cu care lucram în cadrul proiectului. Și nu numai eu aveam această problemă. Și colegii mei din departament se confruntau cu acest inconvenient. Deseori interveneau probleme legate de rezervarea spațiilor de lucru,  dificultăți care se extindeau și în zona locurilor de parcare, ducând la un disconfort și o grijă inutilă pentru a începe o nouă zi de muncă.

\subsection{Scop}
Scopul OfOps este de a optimiza și de a utiliza  eficient resursele comune existente la nivelul firmei, iar motivația realizării aplicației vizează îmbunătățirea desfășurării activităților angajaților la birou creând un mediu propice atingerii obiectivelor firmei.

\subsection{Obiective}
Obiectivele pe care și le propune aplicația OfOps să le îndeplinească sunt:
    \begin{itemize}[left=1.5cm]
         \item Rezervarea optimă a birourilor și a locurilor de parcare, fără ca rezervările să se suprapună cu altele deja existente;
        \item Eficientizarea utilizării sălilor de ședințe
        \item Sugerarea unei alternative în cazul în care sala este ocupată în intervalul dorit;
        \item Utilizarea hărților interactive pentru a face mai ușoară experiența utilizatorului.
    \end{itemize}


\section{Structura lucrării}

