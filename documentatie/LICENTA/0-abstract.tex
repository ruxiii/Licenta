\begin{abstractpage}

\begin{abstract}{romanian}
\hspace{1cm} Aplicația WEB prezentată în această lucrare de licență se numește OfOps, aceasta având ca scop gestionarea spațiilor comune ale unei firme, astfel încât desfășurarea activităților de birou să fie optime.

Locurile uneori insuficiente pentru toți membrii echipei, goana de a ajunge mai devreme să prinzi biroul preferat sau chiar un loc de parcare sunt experiențe neplăcute, trăite, care au reprezentat baza acestei aplicații. Necesitatatea de a ști disponibilitățile birourilor, sălilor de ședință și al locurilor de parcare mi s-a părut un punct de plecare important pentru ca aplicația OfOps să ne scape de grijile venirii la birou.

Aplicația dispune de un calendar și un ceas pentru a selecta momentul în care se dorește o vizualizare a disponibilității în acel moment, dar și de hărți interactive. Acestea au rolul să faciliteze experiența navigării și, tot pe această hartă, se pot observa locurile colorate diferit în funcție de statusul rezervării lor. Paleta de culori folosită pentru a marca locurile este: verde – locul este liber –, roșu – ocupat –  și, în cele din urmă, galben – rezervarea a fost făcută de utilizatorul autentificat în aplicație –.

Partea de rezervare se poate efectua printr-un simplu click pe locul dorit. Însă, în cazul în care mai există deja rezervări pentru acel loc, utilizatorul va fi anunțat printr-o notificare și nu va putea să efectueze rezervarea în cazul în care în intervalul dorit este programată altă rezervare. Pentru sălile de ședință, în schimb, user-ul va primi o recomandare de sală disponibilă pentru intervalul dorit în cazul în care sala nu este disponibilă în intervalul inițial.

Această lucrare va prezenta, în cele ce urmează, toate etapele care au stat la baza dezvoltării aplicației de la gândirea flow-ului inițial și baza sa de date până la API-urile folosite pentru integrarea funcționalităților sale.
\end{abstract}
\end{abstractpage}

\begin{abstractpage}

\begin{abstract}{english}
\hspace{1cm}The WEB application presented in this bachelor's thesis is called OfOps, its purpose being the management of shared spaces within a company, so that office activities can be optimized.

The sometimes insufficient seating spaces for all team members, the rush to arrive early to catch the preferred desk or even a parking lot are unpleasant experiences that have served as the basis for this application. The need to know the availability of desks, meeting rooms, and parking lots was an important starting point for me to develop OfOps in order to relieve us of the worries of going to the office.

The application features a calendar and a clock to select the desired time of availability, as well as interactive maps. These maps are designed to enhance the navigation experience and, on these maps, places are color-coded depending on their reservation status. The color palette used to mark them is: green – available place –, red – occupied – and finally, yellow – reservation has been made by the authenticated user–.

The booking process can be done with a simple click on the desired spot. However, if there is already another reservation for that spot, the user will be notified about it and will not be able to make it if another reservation is scheduled for the wanted interval. For meeting rooms, on the other hand, the user will receive a recommendation for an available meeting room for the desired time if the room is not available in the initial interval.

This paper will present all the stages that have underpinned the development of the application, from the initial flow thinking and its database to the APIs used for integrating its functionalities.


\end{abstract}

\end{abstractpage}