\begin{abstractpage}

\vspace{1cm}
\begin{abstract}{romanian}
\hspace{0.4cm} Aplicația WEB prezentată în această lucrare de licență se numește OfOps, aceasta având ca scop gestionarea spațiilor comune ale unei firme astfel încât desfășurarea activităților de birou să fie eficiente.

Spațiile uneori insuficiente pentru toți angajații firmei, necesitatea de a ajunge mai devreme pentru a găsi liber biroul preferat sau chiar un loc de parcare sunt experiențe neplăcute, trăite, care au reprezentat baza dezvoltării acestei aplicații. Dorința de a cunoaște disponibilitățile birourilor, sălilor de ședință și al locurilor de parcare libere mi s-a părut un punct de plecare esențial pentru ca aplicația OfOps să ne scape de unele griji inutile.

Aplicația dispune de un calendar și un ceas pentru a selecta intervalul de timp în care se dorește o vizualizare a disponibilităților spațiilor la un anumit moment, dar și de hărți interactive. Acestea au rolul să faciliteze experiența navigării și să evidențieze spațiile colorate diferit în funcție de statusul rezervării lor. Paleta de culori folosită pentru a marca locurile este: verde – loc liber –, roșu – ocupat –  și galben – rezervarea a fost făcută de utilizatorul autentificat în aplicație –.

Rezervarea se poate efectua printr-un simplu click pe locul dorit. Însă, în situația în care mai există deja rezervări pentru acel loc, utilizatorul va fi anunțat printr-o notificare în cazul în care în intervalul dorit este programată altă rezervare. Pentru sălile de ședință, user-ul va primi o recomandare de sală disponibilă pentru intervalul solicitat în cazul în care sala selectată inițial nu este disponibilă în acel moment.

Această lucrare va prezenta, în cele ce urmează, toate etapele care au stat la baza dezvoltării aplicației de la gândirea flow-ului inițial și baza sa de date până la API-urile folosite pentru integrarea funcționalităților sale.
\end{abstract}
\end{abstractpage}

\begin{abstractpage}

\begin{abstract}{english}
\hspace{0.4cm} The WEB application presented in this bachelor's thesis is called OfOps, its purpose being the management of shared spaces within a company, so that office activities can be optimized.

The sometimes insufficient seating space for all employees, the rush to arrive early to catch the preferred desk or even a parking lot are unpleasant experiences that were the basis of this application. The need to know the availability of desks, meeting rooms, and parking lots was an important starting point for me to develop OfOps in order to relieve us of the unnecessary worries.

The application features a calendar and a clock to select the desired time of availability, as well as interactive maps. These maps are designed to enhance the navigation experience and, on them, places are color-coded depending on their reservation status. The color palette used to mark them is: green – available place –, red – occupied – and finally, yellow – reservation has been made by the authenticated user–.

The booking process can be done with a simple click on the desired spot. However, if there is already another reservation for that place, the user will be notified about it and will not be able to make it. For meeting rooms, on the other hand, the user will receive a recommendation for an available meeting room for the desired time if the wanted meeting room is not available in the initial interval.

This paper will present all the stages that formed the basis of OfOps's development, from the initial design and its database structure to the APIs used for integrating its functionalities.


\end{abstract}

\end{abstractpage}